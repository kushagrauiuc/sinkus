\documentclass [letterpaper,10pt]{article}
\usepackage[margin=1.3cm]{geometry}
\usepackage{colortbl}
\usepackage{sectsty}
\usepackage{amsmath, amsthm, amssymb}
\usepackage{multicol}
\usepackage{multirow}
\usepackage{graphicx}
\usepackage{soul}
\usepackage[absolute]{textpos}
\definecolor{lightgray}{gray}{0.85}
\sethlcolor{lightgray}
\renewcommand{\rmdefault}{ptm}
\frenchspacing \pagestyle{empty}
\sectionfont{ \sectionrule{0pt}{0pt}{-5pt}{1pt} }
\newcommand{\NewPart}[1]{\section*{\large\textsc{#1}}}
\begin{document}
%\begin{textblock}{5}( 0.6 , 0.3)
%\includegraphics[scale=0.3]{images.jpg}
%\end{textblock}
\begin{textblock}{5}( 6 , 1)
\begin{flushleft}\textbf{{\LARGE \textsc{Kushagra Singhal}}}\end{flushleft}
%Department of Electrical Engineering\\[0pt]
%Indian Institute of Technology Kanpur\\[0pt]
%Specialization: SP/Com}\end{flushleft}
\end{textblock}
\begin{textblock}{5}(12,1)
\begin{flushleft}\textbf{kushagraiitk@gmail.com\\[0pt]
+917060989669}\end{flushleft}
\end{textblock}
\begin{textblock}{5}(0.9,1)
	\begin{flushleft}\textbf{F-93 Kamla Nagar,\\[0pt]
			Agra, UP, 282005}\end{flushleft}
\end{textblock}
.
\vspace{25 pt}
\NewPart{Education}{}
\begin{tabular}{@{}p{15.5 cm}r}
	\bf{University of Illinois at Urbana-Champaign (UIUC)} & \hfill \textbf{December 2016} \\
	\begin{tabular}{p{2 cm} l r}
		\textit{Degree} &  Master of Science in Electrical and Computer Engineering & \textit{CGPA} : 3.92/4.00 \\ 
	\end{tabular}   \\
\end{tabular}\\
\begin{tabular}{@{}p{16 cm}r}
	\bf{Indian Institute of Technology (IIT) Kanpur} &\hfil \textbf{June 2014} \\
	\begin{tabular}{p{2 cm} l r}
		\textit{Degree} &  B.Tech. and M.Tech. ( Dual Degree ) in Electrical Engineering &      \textit{CGPA} : 9.0/10.0 \& 10.0/10.0 \\
	\end{tabular}   
\end{tabular} 
\vspace{-10pt}
\NewPart{Work Experience}{}
%\vspace{-3pt}
%\textbf{Analyst (GT), Flipkart.com, Bangalore, India} \hfill\emph{(June - July 2014)}\\
%\vspace{-16pt}
%\begin{itemize} \itemsep -2.5pt
%	\item Conceptualized a machine learning solution to optimize warehouse space utilization for apparels category
%	\item Extracted features from historical warehouse data to cluster products into stocking categories and forecast high demand times 
%\end{itemize}
\vspace{-3pt}
\textbf{Associate ( MRM\footnote{Model Risk Management} Group ), Goldman Sachs, Bangalore, India} \hfill\emph{(January 2019 - Present)}\\
\vspace{-16pt}
\begin{itemize} \itemsep -2.5pt
	\item Overseeing the validation of all stress test models ( credit, mortgages ) used for risk management and regulatory submissions
	\item Designed a model to forecast P\&L profile of credit cards (Apple Card) in financial crisis scenarios using FRED delinquency data
	\item Developed a model to calibrate price and option adjusted spread time series for mortgage loans using Intex securitization data
	
\end{itemize}
\vspace{-3pt}
\textbf{Senior Analyst ( MRM Group ), Goldman Sachs, Bangalore, India} \hfill\emph{(January 2017 - December 2018)}\\
\vspace{-16pt}
\begin{itemize} \itemsep -2.5pt
	\item Enhanced the jump-to-default risk model used by the firm to calculate exposures on various reference obligations
	\item Worked on global market stress scenario development and validation for submission to the FRB\footnote{Federal Reserve Board} as a part of the CCAR\footnote{Comprehensive Capital Analysis and Review} process
	\item Developed regression models to calibrate risk factor moves in credit spread widening scenarios used to set desk trading limits
\end{itemize}
%\vspace{-3pt}
%\textbf{Programming Intern, Vrije Universteit Brussels, Belgium} \hfill\emph{(May - July 2012)}\\
%\vspace{-16pt}
%\begin{itemize} \itemsep -2.5pt
%	\item Applied estimation techniques to identify unknown linear system models based on known inputs and noisy outputs
%	\item Assisted doctoral students in implementing system identification algorithms efficiently in MATLAB 
%	\item Instructed undergraduate students at VUB in basics of MATLAB programming
%\end{itemize}
\vspace{-15pt}
\NewPart{Research Experience and Academic Projects}{}
\textbf{Social Network Deanonymization using Side Information} \emph{(Research Assistant, UIUC)}\hfill\emph{(August - December 2016)}\\
\vspace{-16pt}
\begin{itemize} \itemsep -2.5pt
	\item Designed an algorithm to match users across two correlated social graphs using community structure as side information
	\item Proved the algorithm's asymptotic correctness, using tools from probability and graph theory, for matching random graphs 
	\item Highlighted privacy issues in releasing social network datasets by mapping users across large real networks with low error rates
\end{itemize}
\textbf{Enabling Privacy Preserving Data Analytics for Anonymized Networks} \emph{(Research Assistant, UIUC)}\hfill \emph{(August 2014-July 2015)}\\
\vspace{-16pt}
\begin{itemize} \itemsep -2.5pt
	\item Analyzed the problem of preserving community structure without privacy breach while releasing anonymized social network data 
	\item Derived asymptotic converse \textbf{(first in literature)} and improved achievability \textbf{(by a factor of $\mathbf{4}$)} bounds for matching correlated random graphs generated using the Stochastic Block Model
\end{itemize}
\textbf{Two Dimensional Source Localization using Spherical Arrays} \emph{(Dual Degree Thesis, IIT Kanpur)}\hfill\emph{(May 2013 - May 2014)}\\
\vspace{-16pt}
\begin{itemize} \itemsep -2.5pt
	\item Modeled the speech multi-source localization problem in two dimensions using the spherical microphone array
	\item Developed a sparse reconstruction algorithm based on partial dictionaries to estimate sources' elevation and azimuth
	\item Improved resolution probability and demonstrated application to source tracking in noisy environments for different trajectories
\end{itemize}
%\textbf{Data Structure for Dynamic Paths } \emph{(Course Project : Design and Analysis of Algorithms , IIT Kanpur)}\hfill\emph{(January - April 2012)}\\
%\vspace{-15pt}
%\begin{itemize} \itemsep -2.5pt
%	\item Designed an AVL tree based data structure to represent weighted and directed dynamic paths 
%	\item{Developed an efficient algorithm to perform operations (splitting and merging of two directed paths, reversing paths, add	weight) and answer queries (minimum weight edge, reachability) in O($\log n$) time}
%	\item Implemented the algorithm in C for large scale paths and passed highest number of test cases in the class out of over $50$ students
%	\end{itemize}
\vspace{-15pt}
\NewPart{Technical Proficiency}{}
\begin{itemize}\itemsep -2.5pt
	\item Proficient in C, C++, MATLAB, Python, \LaTeX
	\item Familiar with SQL, Octave, HTML, JAVA, Pajek, R, Weka, CVX (Optimization Software)
%	\item Data Structures, Design and Analysis of Algorithms, Probability and Statistics, Statistical Learning Theory, Stochastic Processes, Graph Theory, Linear Algebra, Statistical Signal Processing, Convex Optimization
\end{itemize}
\vspace{-15pt}
\NewPart{Scholastic Achievements}{}
\vspace{-3pt}
\begin{itemize}\itemsep -2.5pt
	\item Awarded {\textbf{Institute Silver Medal}} for graduating top of the Electrical Engineering class at IIT Kanpur
	\item Achieved \textbf{All India Rank 443} (99.9 percentile) in IIT Joint Entrance Examination (JEE) 2009 among 400000 applicants
%	\item Authored $\mathbf{5}$ \textbf{peer reviewed} and $\mathbf{2}$ submitted research papers pertaining to signal processing and social network analysis
	%\item Achieved $\mathbf{99.3}$ \textbf{percentile} in Graduate Aptitude Test in Engineering (GATE) 2013
\end{itemize}
\vspace{-15pt}
\NewPart{Scientific Contributions}{}
\begin{itemize}\itemsep -2.5pt
	\item{ {\bf{K. Singhal}}, N. Kiyavash, ``Significance of Side Information in the Graph Matching Problem'', ArXiv Preprint.}
	\item{D. Cullina, {\bf{K. Singhal}}, N. Kiyavash, P. Mittal, ``On the Simultaneous Preservation of Privacy and Community Structure in Anonymized Networks'', ArXiv Preprint. }
	\item{ {\bf{K. Singhal}}, R. M Hegde, ``A Sparse Reconstruction Method for Speech Source Localization using Partial Dictionaries over a Spherical Microphone Array'', InterSpeech, Singapore, Sep. 2014. }
	\vspace{-7pt}
	%\item L. Kumar; K. Singhal; R. Sinha; R. M. Hegde, ``Significance of the MUSIC-Group Delay Method in an ICA-Beamforming Framework for Speech Separation in Multi Source Environment", NCC-2013.
\end{itemize}



\end{document}